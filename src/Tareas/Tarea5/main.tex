\documentclass{article}
\usepackage[utf8]{inputenc}

\title{Tarea5}
\author{joshuasmp_2001 }
\date{April 2020}

\begin{document}
\begin{enumerate}
\item La respuesta corecta es la d ya que es la unica que da verdadero, porque la e y d son las unicas que compilan pero cuando las ejecutas la e te da falso y la d verdaero.
\item regresa la letra a 
\item manda el error java.lang.ArrayIndexOutOfBounds Execepsion  ya que se sale del rango del arreglo.
\item ocurre lo mismo que en el inciso 3.
\item determina la representación de caracteres para un dígito específico en la raíz especificada. Si el valor de la raíz no es una raíz válida, o el valor del dígito no es un dígito válido en la raíz especificada, se devuelve el carácter nulo ('\ u0000').
\item me devuelve 4 y nada 
\end{enumerate}

\end{document}
