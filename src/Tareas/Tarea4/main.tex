\documentclass{article}
\usepackage[utf8]{inputenc}

\title{Tarea4}
\author{joshuasmp_2001 }
\date{April 2020}

\begin{document}

Montaño Pérez Joshua Said 
317222812


1.


Nos sirve para poder meterle valores a nuestro progrma 
por ejemplo si tenemos un programa que organiza numeros lo que tenemos que hacer es poner java ordenar 1 5 7 0 9 y el array recogera esos datos y luego los acomodara segun nuestro programa.

2.


a)Pattern 
    delimiter()
    devuelve el patrón que este escaner esta utilizando actualmente para hacer coincidir los delimitadores.
    
    
b)String 
    findInLine(Pattern pattern)
    intenta encontrar la proxima aparicion del patron especificado ignorado delimitadores.
    
    
c)String 
    findInLine(String pattern)
    intenta encontrar la proxima aparicion de un patron construido a partir de la decadena especificada, ignorado los delimitadores.


d)boolean 
    hasNextDouble()
    devuelve verdadero si el proximo token en la entrada de este escaner se puede interpretar como un valor double usando el metodo nextDouble().
    
    
e)boolean 
    hasNextInt()
    devuelve verdadero si el siguiente token en la entrada de este escaner se puede interpretar como un valor int en la raiz predeterminada utilizando el metodo nextint():
    
    
3.

Este error se genera porque lo que nos estan poniendo en el scanner no es lo que pedimos, esto quiere decir que el parametro que el programa esta pidiendo no es el mismo o tambien esta la opcion de que el parametro se alla salido del rango.


4.

a) son 4 iteraciones 
el valor final de i es 5 y el alcance de la variable es en todo el programa. 


b) solo es una iteración ,el valor final es de 5 y  la variable alcanza todo el programa 

c) son 3 teraciones, el valor final es de 4 y este tiene un error porque la variable i esta en el for, entonces lo que debemos de hacer es crear afuera la i para que pueda ser utilizada en todo el programa. 


d) son 4 ieraciones para i y 3 iteraciones para j, el valor final de i es 3 y el de j es 2 , por último el alcance de las variabres es en el for por lo que tenemos que declarar i y j afuera del for  para que tengan alncase en todo el programa. 


5.


a)Porque se me hizo más fácil y más práctico hacerlo de la manera que hice. 


b) 

c) Se me ocurre poner una opción de que si introduce letras en los parámetros le mande un mensaje de que el programa solo acepta números. 
 
\end{document}
